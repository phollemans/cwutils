\chapter{Common Tasks}
\label{common}

The first step in using the CoastWatch Utilities is to discover which
tool to use for the task at hand.  To help with this search, this
section categorizes the tools by the most common tasks that data users
perform.  Use this section as a guide to the functionality
of the tools, while referring to the manual pages of \autoref{manual}
for complete details on each tool's behaviour and command line
options.

\section{Information and Statistics}

The \hyperlink{cwinfo}{cwinfo} tool lists data file contents,
including various global file attributes and the full set of
earth data variables in the file.  This tool is mainly useful
because its output is human readable, as opposed to a raw data
dump from a generic HDF tool.  It also allows you to display
latitude and longitude data for the data corners and center
point.  The \longoption{verbose} option shows the process of
identifying the file format, useful for file format debugging or
when you suspect a file is corrupt.

The \hyperlink{cwstats}{cwstats} tool calculates statistics for
each earth data variable in the file: count, valid, minimum,
maximum, mean, standard deviation.  This is good for assessing
the data quality and checking to see that the data values fall
into the expected range.  Use \longoption{sample}=0.01 to sample
only 1\% of the data as this saves a large amount of I/O and
computation time.  The {\em count} is the total number of data
values (rows $\times$ columns), while the {\em valid} value is
the number of data values that were not just fill values.
CoastWatch satellite data does not always cover the full region
of the file, since the satellite view may have clipped the edge
of the region during overpass.  In these cases, fill values are
written for the missing data and the fill values are skipping
during statistics computations.  Fill values are also used to
signify that data has been masked out for quality purposes (see
the \hyperlink{cwmath}{cwmath} tool for an example of masking and
\autoref{processing} for details on how masking can be used).

The \hyperlink{hdatt}{hdatt} tool is only useful for CoastWatch HDF
files\footnote{The hdatt tool works with any HDF 4 file using the
SDS interface, but is primarily intended for CoastWatch HDF files,
as opposed to any other non-HDF file format
supported by the utilities.}, and performs low-level reading and
writing of HDF attribute data.  You can use it to change an attribute
value without rewriting the file, or to append an attribute value
to the file.  The hdatt manual page gives many good examples of
when it may be required to modify or create attributes.

\section{Data Processing}
\label{processing}

The \hyperlink{cwimport}{cwimport} tool converts data into CoastWatch
HDF format.  Note that it is {\em not necessary} to convert all
data into CoastWatch HDF before working with the data using the
CoastWatch Utilities.  In many cases, you can perform the same
operations on the data in its native format, especially when the
operation only reads information and performs no file modification.
This is due to the design of the CoastWatch Utilities, which contains
a software layer (called the {\em Earth Data Reader} or {\em EDR}
layer) separating the tools from the physical input file format.
The \hyperlink{cwexport}{cwexport} tool complements cwimport -- it
converts data into various simple text or binary formats.  The
cwexport tool benefits from the EDR layer as well, and as such can
export data from CoastWatch HDF, NetCDF, NOAA 1b, and
others.

During satellite data processing, it is often necessary to compare data
from the satellite sensor to data from in-situ measurements.  The
\hyperlink{cwsample}{cwsample} tool performs this function by taking
as input a geographic latitude/longitude point or set of points and
printing out the data values found at those points in the file.

Another common task in data processing is to combine data from one
or more variables using a mathematical equation, or to combine data
from multiple files in a data composite.  The \hyperlink{cwmath}{cwmath}
tool takes a math expression of the form $y = f(a,b,c,...)$ where
$a,b,c...$ are earth data variables in the file, and loops over
each pixel in the file to compute $f$.  The
\hyperlink{cwcomposite}{cwcomposite} tool combines data from one
earth data variable across multiple files by computing the mean,
median, minimum, maximum, or latest value.  You can use the cwmath
and cwcomposite tools in tandem to create composite data for a given
earth variable; for example to create a weekly composite of sea-surface
temperature (SST), mask out all cloud contaminated SST from each
file using the cwmath {\em mask} function, and follow by running
cwcomposite on all the masked SST files to compute the mean or
median.

Certain data processing tasks are beyond the capabilities of the
CoastWatch Utilities command line tools.  In such a case, the recommendation
is to:
\begin{enumerate}

\item Access and process data using the Java Language API
outlined in \autoref{api}, either by writing Java code or by using a script 
written in BeanShell (\url{http://beanshell.org}) with the 
\hyperlink{cwscript}{cwscript} tool.

\item Use native C code with the HDF or NetCDF libraries (see \autoref{third}).

\item Use a higher level programming language like IDL, Matlab, or Python 
which have HDF and NetCDF access libraries available.

\end{enumerate}
The advantage of using the Java API provided by the CoastWatch
Utilities is that metadata and file I/O operations are already
implemented and generally transparent to the programmer.

\section{Graphics and Visualization}

The main interactive tool for displaying CoastWatch data files is the
CoastWatch Data Analysis Tool (CDAT).  The complexity of CDAT is such
that it deserves its own section -- see \autoref{cdatchap} for a
complete discussion of CDAT and its features.  CDAT is mainly useful
for creating unique plots of CoastWatch data, performing data
surveys, and drawing annotations on top of the data image.  In
contrast, the command line tools discussed in this section are designed
for the automated creation of plots and graphics from CoastWatch data
from a scripting language such as Unix shell, Perl, or Windows batch
files.

The \hyperlink{cwrender}{cwrender} tool creates images from earth data
variables, using either a palette or three channel composite mode.
Many rendering options are available including coast line, land mask,
grid line, topographic, and ESRI shapefile overlays, custom region
enlargement, and units conversion.  Output formats supported include
PNG, JPEG, GIF, GeoTIFF, and PDF.

The \hyperlink{cwcoverage}{cwcoverage} tool creates
graphics for documentation or web pages that show the physical area
that a data file or group of files covers on the earth.  The output
shows an orthographic map projection with the boundary of each data
file traced onto the earth and labeled.  Along similar lines, the
\hyperlink{cwgraphics}{cwgraphics} tool creates land, coast, and grid 
graphics for the region covered by a data file.  The output of
cwgraphics is inserted into the data file as an 8-bit data variable
for later use by cwrender or alternatively to be exported via cwexport
for use in custom rendering or masking routines.

\section{Registration and Navigation}
\label{registration}

Earth data from two data files is said to be {\em in register} if
every corresponding pair of pixels has the same earth location.  We
use the term {\em registration} to refer to the process of resampling
data to a {\em master} projection.  Data that has first been
registered to a master can be intercompared or combined with
other data registered to the same master.  Standard CoastWatch
map-projected data files that have been registered to the same master
projection can be intercompared or combined on a pixel by pixel basis.

You can create your own master projections and CoastWatch map-projected
data files using the \hyperlink{cwmaster}{cwmaster} and
\hyperlink{cwregister2}{cwregister2} tools.  The cwmaster tool is an
interactive tool for designing master map projections.  The tool
enables you to create CoastWatch HDF masters that use one of the
various map projections supported by the General Cartographic
Transformation Package (GCTP)\footnote{See \autoref{metadata} for
a discussion of CoastWatch HDF metadata which relies on GCTP for
map projection parameters.}, such as Mercator, Polar Stereographic,
Orthographic, and many others.  Once a master is created 
you can used it in the cwregister2 tool to resample data to the new
master projection.

When earth image data is captured from a satellite and processed,
there are cases in which inaccuracies in satellite position and
attitude (roll, pitch, and yaw) cause coastlines to appear {\em
shifted} with respect to the image data.  We say that such data
requires a {\em navigation correction}.  Ideally the navigation
correction should be applied to the satellite data while in the
sensor projection before any registration to a map projection master.
In reality data users often only have access to the final map-projected
products.  However since these map-projected products generally
cover a small geographic area, acceptable navigation corrections
can often be achieved by applying an offset to the image data of a
few pixels in the rows direction and a few pixels in the columns
direction.  You can use the \hyperlink{cwnavigate}{cwnavigate} and
\hyperlink{cwautonav}{cwautonav} tools to apply navigation corrections
to CoastWatch data files.  The cwnavigate tool applies a manual
correction, normally derived from your own observation of the data
or from some preexisting database of corrections.  The cwautonav
tool attempts to derive and apply a correction automatically from
the image data in the file.

A final tool related to registration/navigation is
\hyperlink{cwangles}{cwangles} which computes explicit latitude,
longitude, and solar angles based on metadata in the CoastWatch
data file.  Some data users appreciate having latitude and longitude
values at each pixel rather than simply GCTP map projection parameters
or swath polynomial coefficients.  Such earth location data is
useful when exporting CoastWatch data for use in other software
packages.  Note that text output from the cwexport tool (see
\autoref{processing}) can also include the latitude and longitude
of each pixel without having to run cwangles.

%%\section{Network}
%%
%%The \hyperlink{cwdownload}{cwdownload} tool has a command line only
%%interface and is recommended for advanced data users only.  You can
%%use the tool to retrieve recent data files from a CoastWatch data
%%server, or to maintain an archive of certain data files of interest.
%%For regular data file retrieval, the tool can be run via the Unix
%%cron daemon or Windows Task Scheduler (located under System Tools
%%in newer versions of Windows).  Currently, only AVHRR data is handled
%%by cwdownload.  You should contact the CoastWatch Help Desk
%%(coastwatch.info@noaa.gov) for access to a CoastWatch data server
%%to use with cwdownload.
%%
%%The \hyperlink{cwstatus}{cwstatus} tool shows a graphical view of the
%%status of a CoastWatch data server, and is intended for use by
%%CoastWatch staff only.  The status tool displays the current list of
%%satellite passes, their properties, and a graphical view of the pass
%%footprint on the earth along with a preview image.
