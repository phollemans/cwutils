\chapter{Introduction}

The main goal of the CoastWatch Utilities software is to aid data
users in working with NOAA/NESDIS CoastWatch satellite data. 
CoastWatch data is distributed as individual files in a scientific
data format that is not recognized by standard image viewers, and the
CoastWatch Utilities are useful for manipulating data and creating images
from CoastWatch data for both recreational and scientific
applications.  CoastWatch data files contain:
\begin{enumerate}

  \item Global file attributes that describe the date/time and
  location of the earth data in the file, as well as any relevant data
  processing details.

  \item Earth data as a set of two-dimensional numerical arrays, each
  with a unique name.  These {\em variables} hold the actual
  scientific data and accompanying attributes, such as scaling factor
  and units, that describe how to use the data values.

\end{enumerate}
The CoastWatch Utilities allow users to selectively access and extract
this information in a number of ways.

\section{A Brief History}

The CoastWatch Utilities have been evolving since 1998.  The original
software was capable of working with CoastWatch IMGMAP format data
files, the standard data distribution format (limited to NOAA AVHRR
data) for CoastWatch satellite data until 2003.  The current utilities
are capable of reading HDF, NetCDF 3, NetCDF 4, and NOAA 1b.  Following
is a sketch of the software evolution:
\begin{itemize}

  \item Version 1: 1997.  The first version created CoastWatch IMGMAP
  files from TeraScan Data Format (TDF), a proprietary format from
  SeaSpace Corporation.  It was never publicly released.

  \item Version 2: 1998-2000.  The second version was released to the
  CoastWatch data user community and worked with CoastWatch IMGMAP
  (CWF) files only.  Users could convert CWF files to other formats,
  create GIF images, perform land and cloud masking, create data
  composites, and other related tasks.  

  \item Version 3: 2001-present.  The third version was designed to have
  a more flexible data handling capability, with support for the new
  CoastWatch HDF and NetCDF formats.  The CoastWatch
  Data Analysis Tool (CDAT) was integrated into the package.
  Additional capabilities were added for NOAA 1b, level 2 swath
  style data, automatic navigational correction, ESRI shapefiles, and
  other improvements.

\end{itemize}

\section{Installation Notes}

\subsection{System Requirements}

\begin{description}

  \item[Operating system] The CoastWatch Utilities are currently
  available for Windows, Mac, and Linux.

  \item[Disk space] A minimum of 300 Mb is required for the installed
  software. We recommended at least 1 Gb of space in total to
  allow for downloading and manipulating satellite datasets. More
  disk space may be required for larger datasets.

  \item[Memory] A minimum of 4 Gb is recommended.  More memory
  may be required depending on the number of concurrent processes
  running on the computer.

\end{description}

\subsection{Downloading}

To download, visit the CoastWatch central web site: 
\begin{quote}
  \url{http://coastwatch.noaa.gov}
\end{quote}
and navigate to the {\gui Software \& Utilities} section under
{\gui User Resources}.  Download the package appropriate for your
operating system. To install, read one of the following sections
depending on your system.

\subsection{Installing on Windows}

If upgrading to a new version, there is no need to uninstall the
previous version -- the new installation setup program will handle
the details.  To install the new version, simply double-click the
downloaded installation package. The setup program will install the
software in a user-specified directory, and add entries to the Start
Menu.  The User's Guide (this document) is also added to the Start
Menu.

If you use the command line tools you may want to add the installed
executable directory to the path for easier command line execution.  Under
Windows 10 go the the Windows {\gui Start Menu} and click the
{\gui Settings} gear icon.  In the search box type ``environment'' and select
{\gui Edit environment variables for your account}.  Under {\gui User variables}
edit the {\gui Path} variable and add a new entry, for example
{\tt C:$\backslash$Program Files$\backslash$CoastWatch Utilities$\backslash$bin}.
Other Windows versions have a similar procedure for adding a path.

\subsection{Installing on Mac}

The MacOS installation package is a disk image (DMG) file that
contains support for Intel-based Macs only.  Download the DMG file,
double-click to mount it, and then
double-click the {\gui CoastWatch Utilities Installer} icon.
When upgrading, the installer will automatically uninstall an old version if
it exists.

If you use the command line tools you may want to add the installed
executable directory to your path for easier command line execution.
In a Terminal session, add the following lines to the {\tt \~{ }/.profile}
file:
\begin{quote}
  {\tt export PATH=\$\{PATH\}:"/Applications/CoastWatch Utilities/bin"}
\end{quote}
and then open a new Terminal window for the changes to take effect.

\subsection{Installing on Linux}

The Linux version installs from a tar archive, extracted as follows:
\begin{quote}
  {\file tar -zxf cwutils-3\_x\_x-linux-x86\_64.tar.gz}
\end{quote}
You may also want to add the installed executable directory to
your {\tt \$PATH} environment variable for easier command line
execution, for example:
\begin{quote}
  {\tt setenv PATH \$\{PATH\}:\$HOME/cwutils/bin}~~~(in {\tt \~{ }/.cshrc}) \\
  {\tt export PATH=\$\{PATH\}:\$HOME/cwutils/bin}~~~(in {\tt \~{ }/.bashrc})
\end{quote}

\section{Using the Software}

The CoastWatch Utilities are made up of various individual tools.
Graphical tools allow you to point and click, working with data
interactively; the graphical tools have a built-in help system
with brief details on each key feature.  To complement these,
there are also command line tools that allow you to process data
using a text-only command prompt or scripting language.  Call any
command line tool with the \longoption{help} option to get a
short synopsis of parameters.  See the manual pages of
\autoref{manual} for a complete discussion on the command line
parameters for both graphical and command line tools.

Functionally, the CoastWatch Utilities are designed to meet the data
processing and rendering needs of many different types of data users.
The individual tools have been developed from both in-house
requirements and data user suggestions.  The functionality of the
tools may be divided into several categories based on data processing
task:
\begin{description}

  \item[Information and Statistics] File contents, statistics
  computations on variables (for example min, max, mean, standard
  deviation), direct access to raw file and variable attributes.

  \item[Data Processing] Data format conversions,
  compositing, generic variable math, data sampling.

  \item[Graphics and Visualization] Interactive
  visualization/analysis, batch image rendering, ancillary graphics
  creation such as data coverage maps, grids, coastlines, landmasks.
    
  \item[Registration and Navigation] Resampling of data from one
  projection to another, interactive generation of region masters,
  manual and automatic navigational correction, computation of solar
  and earth location angles.

  \item[Network] Data download and server status.

\end{description}

\section{Third Party Software}
\label{third}

There are a number of other software packages than can
be used to read data from CoastWatch HDF and NetCDF data files.  They are {\em
generic} in the sense that they can read the numerical and text data,
but they generally cannot interpret the metadata conventions used by CoastWatch.
They are well suited to advanced users who want to have more detail on
the HDF or NetCDF data file contents.  You can refer to the CoastWatch
HDF metadata specification of \autoref{metadata} when using third
party software.

The HDF Group is the main source of information
on the HDF format specification and software libraries.  You can
download tools for working with HDF from the main web site:
\begin{quote}
  \url{http://www.hdfgroup.org}
\end{quote} 
Many of the tools are command line, but HDFView is a useful graphical
tool.  Note that since NetCDF 4 is implemented using HDF, NetCDF 4 files can
also be viewed in HDFView.  For NetCDF-specific software, visit the Unidata site:
\begin{quote}
  \url{http://www.unidata.ucar.edu/software/netcdf}
\end{quote}

A number of data analysis programming languages have also been linked to the
HDF and NetCDF libraries including IDL, Matlab, and Python.

