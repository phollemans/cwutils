\section{Hidden command line options}

There are a number of useful options for the command line tools that
are not documented in the tool manual pages themselves.  This is because
they can be used with all tools or some large subset of tools, and they
alter some subtle feature of a tool's internal behaviour for which there
is a reasonable default setting.  However, some users may want to tune
or optimize these defaults.  All of the options are specified using a somewhat odd
syntax with a leading {\file -J} to protect them from being passed to the tool
command line processing engine -- instead they're intercepted and passed to
the Java VM to become part of the Java environment in which the tool runs.  The
options are as follows:
\begin{description}

  \item[{\file -J-XmxSIZEm}] The maximum Java heap size where SIZE is replaced
  by a size in megabytes, for example {\file -J-Xmx1024m} would specify a
  gigabyte.  The heap is where Java allocates memory dynamically as new objects
  are created on-the-fly.  A larger heap may be required if an out of memory
  error is thrown when running a tool.  The tools all have slightly different
  default values for the maximum heap size, as required by their function.
  For example, cwrender has a default heap size that is larger than cwinfo.

  \item[{\file -J-Dcw.cache.size=SIZE}] The maximum cache size for storing
  tiles of 2D data read from data files, currently only implemented for NetCDF
  files.  The SIZE is replaced by a size in megabytes, with a default value
  of 512 Mb.  For example {\file -J-Dcw.cache.size=1024} specifies a cache
  size of 1024 Mb.  A larger cache size may be needed for some files when the
  files are written with monolithic compression, ie: the entire data variable
  is compressed and written as one chunk which expands to a size larger than
  512 Mb in memory.  We generally discourage data providers from using monolithic data
  variables because they require decompression of an entire variable even if
  only one value or a small section is being accessed, which greatly increases
  the time for interactive display and analysis or automated data processing.

  \item[{\file -J-Dcw.compress.mode=true$|$false}] The compression mode flag for
  writing CoastWatch HDF data files, by default true.  When true, HDF4 data
  files are created or modified so that each variable is compressed into
  square chunks.  In some cases this may be undesirable if processing speed
  is important, for example if the file is being written from cwmath
  as an intermediate step in a longer computation, and the file will be
  deleted afterwards.  When false, HDF4 data files are written without
  chunking or compression.  This increases the speed at which files can be
  written and subsequently read from.

  \item[{\file -J-Dcw.chunk.size=SIZE}] The chunk size used for writing
  CoastWatch HDF data files, by default 512 kb.  The SIZE is replaced by
  a size in kilobytes, so for example {\file -J-Dcw.chunk.size=1024} writes
  data in 1 Mb chunks.  Note that this is different from the chunk dimensions.
  If the chunk size is 512 kb, then 16-bit integer data is written in
  chunks of dimensions 512$\times$512 because 2 bytes per value $\times$ 512
  $\times$ 512 $=$ 524288 bytes $=$ 512 kb.  This is no coincidence, since
  CoastWatch HDF with AVHRR data originally contained mostly 16-bit scaled
  integer data and 512$\times$512 was a convenient chunk size for fast
  reading and display.  Using 512 kb chunks means that 8-bit byte data is written
  in 724$\times$724 chunks, 32-bit float data is written in 362$\times$362
  chunks, and so on.

\end{description}
