\chapter{Programmer's API}
\label{api}

\section{How to use the API}

Developers may want to use some of the same code used by the
CoastWatch Utilities to read and write data, render images, or
perform some variation on the existing functionality.  Since the
CoastWatch Utilities are written almost entirely in Java
(conforming to the Java 5 language spec), the Java Application
Programming Interface (API) documentation is available in the
standard Javadoc-generated HTML web page format, included with the
CoastWatch Utilities distribution package as the compressed
{\file doc/api.zip} file in the installation directory.  Users of
the graphical installers on Windows, Mac, and some Unix versions
have to manually check off the ``Source code and API docs'' when
the package is installed in order to get the API zip file (you
also get the full source code {\file src.zip} file whose Java code
files serve as examples of using the API).  Unzipping the API
file creates a new directory with an {\file index.html} file to
use as the starting point for all API documentation.  The Javadoc
is fairly verbose and can be used as a reference for all Java
classes in the software, along with this chapter which provides a
high-level overview of the main classes.

In order to effectively develop your own Java code that uses the
CoastWatch API and link it to the CoastWatch libraries, you
should be familiar with using the Java compiler, JAR files, how
to navigate through Javadoc pages, and possibly be familiar with
automated code compiling tools such as Ant.  There are a number
of directories and files in the installation that are of use to
developers:
\begin{description}

\item[{\file lib/}] -- Contains the main {\file cwf.jar} file of
CoastWatch Utilities code, plus all JAR files required by the
CoastWatch code.  The major dependencies are as follows:
\begin{itemize}

\item Alloy Look and Feel (used for Unix GUI only):\\
{\file alloy.jar}

\item GeoTools (shapefile support):\\
{\file geoapi-1.1.0alpha.jar}\\
{\file gt2-main-nojai.jar}\\
{\file gt2-shapefile.jar}\\
{\file JTS-1.4.jar}

\item GIF encoding: \\
{\file neuquant.jar} (neural net color quantization) \\
{\file gifencoder.jar} (actual format encoder) \\
{\file kd.jar} (KD-tree color quantization)

\item PDF encoding: \\
{\file itext-1.2.jar}

\item XML parsing: \\
{\file jaxen-full.jar} \\
{\file saxpath.jar}

\item C-style print formatting:\\
{\file hb16.zip}

\item Command line argument parsing:\\
{\file jargs.jar}

\item Java Advanced Imaging CODEC (mainly for GeoTIFF encoding):\\
{\file jai\_codec.jar}

\item Mathematical expression parsing:\\
{\file jep-2.24.jar}

\item Plotting symbols:\\
{\file PlotPackage.jar}

\item Java matrix manipulation:\\
{\file Jama-1.0.1.jar}

\item HDF native library stubs:\\
{\file jhdf.jar}

\item Unidata UDUNITS support:\\
{\file udunits.jar}

\item NetCDF Java support:\\
{\file bzip2.jar}\\
{\file commons-httpclient-2.0.1.jar}\\
{\file commons-logging.jar}\\
{\file dods-1.1.7.jar}\\
{\file gnu-regexp-1.1.4.jar}\\
{\file grib.jar}\\
{\file jdom-b9.jar}\\
{\file jpeg2000.jar}\\
{\file netcdf-2.2.jar}\\
{\file visad.jar}

\item Terrenus HRPT interfaces for NOAA1b instrument data:\\
{\file hrpt.jar}

\end{itemize}

\item[{\file lib/aix},{\file linux},{\file macosx},{\file solaris},{\file
win32}/] -- Native libraries for various operating systems.  C
language code has been compiled to native binary libraries and
linked to Java via JNI in cases where no Java libraries existed.
Native libraries include HDF 4 file I/O, CoastWatch IMGMAP (.cwf)
file I/O, and GCTP map projection coordinate transformations.

\end{description}

To use the API, you {\em must} have the {\file cwf.jar} file in
your Java class path to compile and run custom code, for example
on Unix:
\begin{verbatim}
  javac -classpath ~/cwf/lib/cwf.jar MyCode.java
  java -classpath  .:~/cwf/lib/cwf.jar MyCode
\end{verbatim}
Depending on what CoastWatch classes are used in the
{\file MyCode.java} code, other JAR files as listed above may also
need to be in the Java class path, and as well native library
directories may need to be in the dynamic link path.  For example
using a Unix bash shell under Mac OS X, this code fragment sets
up the environment for compiling and running with the CoastWatch
Utilities:
\begin{verbatim}
  cwfroot="/Applications/CoastWatch Utilities 3.2.2"
  CLASSPATH="."
  for fname in "$cwfroot"/lib/*.jar "$cwfroot"/lib/*.zip ; do 
    CLASSPATH=${CLASSPATH}:"$fname"
  done
  export CLASSPATH
  export DYLD_LIBRARY_PATH="$cwfroot"/lib/macosx
\end{verbatim}
If you prefer to use a GUI development environment such as
Eclipse or JBuilder, then the GUI will have settings to store the
class and native library paths for the current project.
CoastWatch Utilities development is currently performed using Ant
for the build/test cycle and ej-technologies install4j product
(see \url{www.ej-technologies.com}) for building multi-platform
installation packages and launching the various Java programs.

The rest of this chapter is dedicated to describing the main
packages available in the API as follows:
\begin{description}

\item[{\java noaa.coastwatch.io}] -- I/O classes for satellite and
other geographic data formats.

\item[{\java noaa.coastwatch.render}] -- Utility classes for
transforming geographic data into images.

\item[{\java noaa.coastwatch.gui}] -- Graphical classes for custom
user interface components.

\item[{\java noaa.coastwatch.util}] -- General utility classes for
working with geographic data arrays and performing numerical
operations.

\item[{\java noaa.coastwatch.tools}] -- Main program classes for
all GUI and command line tools.

\end{description}

\section{Data I/O}

The {\java noaa.coastwatch.io} package contains a number of classes for
performing I/O with satellite and geographic data formats.  The
superclasses and static method classes for most of the functionality
are as follows:
\begin{description}

\item[{\java EarthDataReader}]~\\ Reads various formats of
geographic data.  The data reader classes read global information
about the data file such as the date and time of the data, what
sensor or model it came from, what organization
gathered/processed the data, and how to transform data array
coordinates into geographic locations on the earth.  The reader
classes also report back on a list of n-dimensional ``variables''
that hold numerical data.  The {\java CWFReader}, {\java HDFReader},
{\java NOAA1bReader}, and {\java OpendapReader} are all examples of
{\java EarthDataReader} classes.

\item[{\java EarthDataWriter}]~\\ Writes various formats of
geographic data.  Just as the data reader classes read global
information and variables, the data writers construct from some
global information and set of variables, and then write out the
numerical data to a specific format.  The {\java BinaryWriter},
{\java HDFWriter}, and {\java TextWriter} are all examples of {\java
EarthDataWriter} classes.

\item[{\java EarthImageWriter}]~\\ Writes various color image
formats of geographic data.  Unlike the data writer classes, the
{\java EarthImageWriter} (i) has no subclasses and (ii) only writes
color pixel data to an output file rather than numerical data.
Image formats include PNG, JPEG, GeoTIFF, GIF, and PDF.  The {\java
EarthImageWriter} makes use of {\java GeoTIFFWriter}, {\java
GIFWriter}, and {\java WorldFileWriter} to perform parts of its
job.

\item[{\java CachedGrid}]~\\ Reads 2D numerical data using a
least-recently-used caching strategy.  Data format specific
subclasses of {\java CachedGrid} form the basis of most of the
high-performance I/O operation in the CoastWatch Utilities.  {\java
CWFCachedGrid}, {\java HDFCachedGrid}, and {\java NOAA1bCachedGrid}
are all subclasses.  The {\java OpendapGrid} class is not part of
the {\java CachedGrid} hierarchy but performs some similar
operations for OPeNDAP data sources.

\item[{\java EarthDataReaderFactory}]~\\ Perhaps one of the most
useful classes, the reader factory has one static method, {\java
create()}, that takes a data file name, automatically identifies
the file format, and returns an {\java EarthDataReader} object.
The reader factory has a special property: it can be extended to
read data formats that are not supported by the standard
CoastWatch Utilities distribution.  To add your own data format
support to CDAT and the CoastWatch Utilities command line tools,
simply subclass {\java EarthDataReader}, place the compiled code
into a JAR file in the {\file lib/} directory, and add the name
of your subclass to the {\file extensions/reader.properties}
file.  Most tools in the package use {\java
EarthDataReaderFactory.create()} for data reading, so your custom
data format will be easy to incorporate into most tool
operations.  In case of problems with automatic file
identification, use the \hyperlink{cwinfo}{cwinfo} tool with
\shortoption{v} option to print verbose messages during the file
identification process.

\end{description}

\section{Data rendering}

The {\java noaa.coastwatch.render} package is the heart of all
rendering operations: converting 2D numerical geographic data
arrays to a color image (palettes, enhancement functions),
drawing line, symbol, and text overlays, and drawing legends
(logos, icons, text, color scales) The main classes are as
follows:
\begin{description}

\item[{\java EarthDataView}]~\\ Renders 2D geographic data to a
color image.  The data view classes take numerical data and
convert it to colors using various schemes: {\java
ColorEnhancement} uses a color palette and enhancement function
where as {\java ColorComposite} uses three enhancement functions,
one each for the red, green and blue components of the output
color.

\item[{\java EarthDataOverlay}]~\\ Renders line, mask, symbol,
and text overlays for an {\java EarthDataView} object.  There are
more than a dozen concrete overlay classes for all different
purposes: {\java MaskOverlay}, {\java TextOverlay}, {\java
CoastOverlay}, {\java TopographyOverlay}, {\java LatLonOverlay},
and so on.  Each overlay class draws graphics to the view by
implementing the {\java render()} method and in some cases uses
only the data view properties and other cases uses supplementary
data such as coastline segment data or topographic elevation
data.

\item[{\java Legend}]~\\ Renders legends for plots with
information such as color scaling, date and time of the data,
geographic location and projection.  The two classes of legends
are {\java DataColorScale} and {\java EarthPlotInfo}, mainly to
accompany the output from an {\java EarthDataView}.

\item[{\java Feature}]~\\ Stores the geometry of some geographic
feature, along with a set of attributes attached to the feature.
The concrete feature classes are {\java PointFeature}, {\java
LineFeature}, and {\java PolygonFeature} to handle those major
feature types.

\item[{\java FeatureSource}]~\\ Reads or creates a set of
features from some geographic data source.  There are many
concrete feature sources for various data including {\java
BinnedGSHHSReader} for reading GSHHS coastline polygon data,
{\java ContourGenerator} for generating contour line features
from numerical data, and inner classes of {\java
ESRIShapefileReader} for extracting features from ESRI shapefile
format files.

\item[{\java PointFeatureSymbol}]~\\ Draws symbol graphics, mainly
for use with an {\java EarthDataView} in the context of a {\java
PointFeatureOverlay}.  The {\java ArrowSymbol} and {\java
WindBarbSymbol} are examples of concrete symbol classes for
drawing vector wind and current fields.

\item[{\java EnhancementFunction}]~\\ Specifies the functional
form for mapping data values to a normalized range of $[0..1]$
for use with a {\java Palette} in a {\java ColorEnhancement}.
{\java LinearEnhancement}, {\java StepEnhancement}, and {\java
LogEnhancement} are all types of functions.  The {\java
EnhancementFunctionFactory} is useful for creating enhancement
functions from a text specification and range.

\item[{\java Palette}]~\\ Holds a set of colors for use in a
{\java ColorEnhancement}.  The {\java PaletteFactory} is useful
for obtaining one of many predefined palettes, mostly derived
from IDL but also some special-purpose CoastWatch palettes.

\end{description}

\section{Graphical user interface components}

The {\java noaa.coastwatch.gui} package is the largest and most
complex package in the CoastWatch Utilities, and provides all the
GUI components for the \hyperlink{cdatchap}{CoastWatch Data
Analysis Tool}, \hyperlink{cwmaster}{CoastWatch Master Tool
(cwmaster)}, and \hyperlink{cwstatus}{CoastWatch Status Tool
(cwstatus)} using Java Swing for platform-dependent look and
feel.  There are far too many classes to describe here, but the
following are some of the more general and re-usable classes:
\begin{description}

\item[{\java HTMLPanel}]~\\ Displays an HTML web page and allows
for hyperlinks and page navigation (forward, back, top, etc).
The panel is the basis of the {\gui Help and Support} menus in
the graphical tools.

\item[{\java EarthDataViewController}]~\\ Displays the contents
of a {\java noaa.coastwatch.io.EarthDataReader} object in a panel
and provides various other control panels for users to control
the data view.  The controller is the basis for each tab in CDAT.

\item[{\java MapProjectionChooser}]~\\ Lets you choose the
specifications of a GCTP-style map projection.  This panel is
used in the CoastWatch Master Tool.

\item[{\java EarthDataChooser}]~\\ A panel and dialog for
choosing {\java noaa.coastwatch.io.EarthDataReader} objects using
either network or local file access and showing a variable
preview image.

\item[{\java LightTable}]~\\ A general purpose drawing class that
displays rubber band style graphics on top of another component
for use in interactive drawing input.  This class is responsible
for the rubber banding in CDAT during a zoom operation, or while
drawing annotations and surveys.

\item[{\java PaletteChooser}]~\\ Presents a list of {\java
noaa.coastwatch.render.Palette} objects and lets you select one
of them, showing a preview of the selected palette.

\item[{\java FullScreenWindow}]~\\ Displays a component in a full
screen window with a toolbar.  This class is used in CDAT for
displaying the data view in full screen mode.

\end{description}

\section{General utility classes}

The {\java noaa.coastwatch.util} package provides a number of
general utility classes for working with geographic data:
\begin{description}

\item[{\java EarthLocation}]~\\ Holds a latitude and longitude
value and allows for intelligently incrementing the values,
formatting them to standard string representations, shifting
datums, and computing the physical distance to other locations.

\item[{\java DataLocation}]~\\ Holds an n-dimensional index for
use in addressing a value in an {\java DataVariable}.

\item[{\java EarthTransform}]~\\ Converts {\java DataLocation}
objects to and from {\java EarthLocation} objects in order to tie
data to a physical geographic space.  {\java MapProjection},
{\java SwathProjection}, and {\java SensorScanProjection} are all
examples of transforms.

\item[{\java Function}]~\\ Maps a set of variables to a function
value such that $y = f(x_1, x_2, x_3, ...)$.  {\java
LagrangeInterpolator}, {\java BivariateEstimator}, and {\java
noaa.coastwatch.render.EnhancementFunction} are examples of
concrete {\java Function} classes.

\item[{\java DataVariable}]~\\ Holds an n-dimensional array of
data and allows for get/set of values, units conversion, value
formatting, and statistics.  The variable has two concrete
classes with specialized methods for their rank: {\java Line} for
1D data, and {\java Grid} for 2D data.

\item[{\java EarthDataInfo}]~\\ Holds information about a
geographic dataset including the data source, the date and
duration of data recording (possible more than one), and the
{\java EarthTransform} object.

\item[{\java LandMask}]~\\ Consults a land mask database to
return true or false for the query: ``Is there land at location
(lat,lon)?''.

\item[{\java Statistics}]~\\ Computes statistics for a set of
data values including the minimum, maximum, mean, standard
deviation, median, and average deviation.  This class is used for
all statistics computations in the CoastWatch Utilities.

\item[{\java GridResampler}]~\\ Computes a resampling of 2D
{\java Grid} data from one {\java EarthTransform} to another.
There are two algorithms for resampling used by the concrete
{\java InverseGridResampler} and {\java MixedGridResampler}
classes.

\end{description}

\section{Main programs}

The main program classes in the {\java noaa.coastwatch.tools}
package include all the command line tools, graphical tools, and
some supporting classes for preferences and user resources.  The
main programs' source code are a reference for using the
top-level API classes. See \autoref{manual} for the full manual
pages for all main programs, and \autoref{common} for a
category-based guide to using the programs.
